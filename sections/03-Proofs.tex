\documentclass[../main.tex]
		
\begin{document}
\section{Proofs}
\subsection{Intro}
\begin{description}
    \item[Theorem: ]Always omits a universal quantifier when it states. Its proofs often have "obviously"/"clearly' indicating that stpes have been omitted the author expects the reader to be able to fill in without hints. 
            
\end{description}
\subsubsection{Direct Proofs}
    For conditional statement $p\rightarrow q$. If $p$ is true, $q$ cannot be false.
\begin{description}
    \item[Def1: ] $\exists k\in\mathbb{Z},n=2k$ and $n$ is \emph{even}. $n$ is odd if $n=2k+1$. No integer between even or odd. 
\end{description}
\subsubsection{Proof by Contraposition}
To prove $p\rightarrow q$ is equal to prove $\lnot q\rightarrow \lnot p$.
\subsubsection{Vacuous and Trivial Proofs}

Vacuous: $p\rightarrow q$ is always true if $p$ is false.
Trivial: $p\rightarrow q$ is true if $q$ is true already.

\begin{description}
    \item[Def2: ] Rational number r. $r\in \mathbb{R}$ if $\exists p,q\in \mathbb{Z}$ where $q\neq0,s.t.~r=\frac{p}{q}$. A number is \emph{irrational} which isn't a rational.
\end{description}
\subsubsection{Proofs by Contradiction}
(omitted)
\subsection{Method and Strategy}
\subsubsection{Exhaustive Proof}
Those proofs proceed by exhausting all possibilities(aka. Proof by Covering all Cases).



			
	

\end{document}





% \begin{description}
%     \item[Theorem: ]Always omits a universal quantifier when it states. Its proofs often have "obviously"/"clearly' indicating that stpes have been omitted the author expects the reader to be able to fill in without hints. 
% \end{description}
% \subsection{Direct Proofs}

% \begin{description}
%     For conditional statement $p\rightarrow q$. If $p$ is true, $q$ cannot be false.
%     \item[Def1: ]$\exists k\in\mathbb{Z}$ 
% \end{description}


